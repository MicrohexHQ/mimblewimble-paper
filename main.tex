% Default course lecture note template by asp 
\documentclass[letterpaper]{article}
\usepackage[top=3cm, bottom=3cm, left=3.85cm, right=3.85cm]{geometry}
\usepackage[onehalfspacing]{setspace}
\usepackage{mathptmx}
\usepackage{amsmath,amssymb,wasysym,amsthm}
\usepackage{hyperref}
\usepackage{lineno,gitinfo}

\newtheorem{lmma}{Lemma}
\newtheorem{corl}{Corallary}
\newtheorem{thrm}{Theorem}
\newtheorem{defn}{Definition}
\newtheorem{expl}{Example}

\newcommand{\Setup}{\mathrm{Setup}}
\newcommand{\Sign}{\mathrm{Sign}}
\newcommand{\Verify}{\mathrm{Verify}}
\newcommand{\Commit}{\mathrm{Commit}}
\newcommand{\Open}{\mathrm{Open}}
\newcommand{\sk}{\mathrm{sk}}
\newcommand{\pk}{\mathrm{pk}}
\newcommand{\e}{\hat{e}}
\newcommand{\hash}{\mathcal{H}_2}
\newcommand{\truth}{\{\mathrm{true}, \mathrm{false}\}}
\newcommand{\randgets}{\xleftarrow{\$}}

\title{Mimblewimble}
\author{Andrew Poelstra\footnote{\texttt{grindelwald@wpsoftware.net}}}
\date{\gitAuthorDate{} (commit \texttt{\gitAbbrevHash)}}

\begin{document}

\maketitle

\begin{abstract} 
\textbf{This is badly incomplete and maybe some proofs are wrong. If you have
encountered this paper floating around the Internet, please disregard it.
Sorry to be a tease. Full version forthcoming.}

At about 04:30 UTC on the morning of August 2nd, 2016, an anonymous person
using the name Tom Elvis Jedusor signed onto a Bitcoin research IRC channel,
dropped a document hosted on a Tor hidden service\cite{voldemort2016}, then signed out. The
document, titled Mimblewimble, described a blockchain with a radically
different approach to transaction construction from Bitcoin, supporting
noninteractive merging and cut-through of transactions, confidential
transactions, and full verification of the current chainstate without
requiring new users to verify the full history of any coins.

Unfortunately, while the paper was detailed enough to comminicate its main
idea, it contained no arguments for security, and even one mistake CITE CITE.
The purpose of this paper is to make precise the original idea, and add
further scaling improvements developed by the author.
\end{abstract}

\paragraph{License.} This work is released into the public domain.

\clearpage
\tableofcontents

\modulolinenumbers[10]
\linenumbers

\section{Introduction}

In 2009, Satoshi Nakamoto introduced the Bitcoin cryptocurrency\cite{nakamoto2009},
an online currency system to allow peer-to-peer transfer of digital tokens.
These tokens' ownership is ascertained by the use of public keys, and transfer
is accomplished by means of a public ledger, a globally replicated list of
transactions which destroy \emph{unspent transaction outputs} (UTXOs) and
create new ones of equal value but different keys.

All Bitcoin coins originate in \emph{coinbase transactions}, transactions which
create new outputs without destroying any old ones, and which are limited in
value to maintain a fixed inflation schedule. Once created, they change hands
by means of ordinary transactions. This means that new users, in order to
fully verify that their view of the system is uncompromised (absent theft or
illegal inflation), must download and validate the entire history from the
system's genesis in 2009 until today. In other words, they must download
and ``replay'' every transaction that has ever occurred.

Today the Bitcoin blockchain sits just shy of 100Gb on the author's disk.
Had Bitcoin used Confidential Transactions\cite{maxwell2015} (CT), each of the
approximately 430 million outputs would consume 2.5Kb each, totalling over
a terabyte of historical data\footnote{The author is aware of unpublished
research that would reduce this quantity by about 25\%; the total is still
formidable.}.

In this paper, we describe a research project to design a Bitcoin-like
cryptocurrency, which achieves dramatically better scaling and privacy
properties than Bitcoin, though at cost of sacrificing functionality.

\subsection{Overview and Goals}

Mimblewimble is a design for a cryptocurrency whose history can be
compacted and quickly verified with trivial computing hardware even
after many years of chain operation. As a secondary goal, it should
support strong user privacy by means of confidential transactions
and an obfuscated transaction graph. However, to achieve these goals,
Mimblewimble cannot support a general-purpose scripting system such
as that in Bitcoin. This precludes such functionality as zero-knowledge
contingent payments\cite{maxwell2016}, cross-chain atomic swaps\cite{nolan2013}
and micropayment channels\cite{poon+dryja2016}. Further research is
needed to emulate these functionalities on top of Mimblewimble, and
is discussed in Section \ref{sec:ext}.

More precisely,
\begin{itemize}
\item Direct transfer of value between parti(es) to other parti(es)
should be possible on the system.

\item All transactions should use confidential transactions
to blind their output amounts.

\item Transactions should be noninteractively aggregable\cite{mouton2013},
such that parties not privy to the original transactions are unable to
separate them. This can improve censorship resistance and privacy,
though it is unclear how to design a safe peer-to-peer network capable
of exploiting this ability.

\item For a new participant, the amount of bandwidth and processing power
needed to catch up with the system should be proportional to the \emph{current
state} of the system, \emph{i.e.} the size of the UTXO set, rather than
the total size of all historical transactions.

As described in the next section, and more thoroughly in Section
\ref{sec:consensus}, Mimblewimble has a scheme for compressing blockchain
history to a size polylog in the original size, which in practice for a
system with Bitcoin's scale should allow a decade or more of transaction
history to be verified in several seconds using a couple megabytes of data\footnote{Experiments
with the compact chains described in this paper suggest that a 500000
block chain may have a compact length of around 300 blocks (though with
high variance across multiple experiments), each containing sinking
signatures which average ten 96-byte elliptic curve points, merkle
commitments to previous blocks consisting of forty 40-byte (blockhash,
difficulty) pairs, and some extra block header data, totalling less
than 3Kb of data. Across 300 blocks this is 900Kb total of compact
chain data. Verification time is dominated by pairing computations,
of which there are on average ten per block, or 3000. On the author's
laptop using Ben Lynn's \texttt{libpbc} library, these can be done
in under 20 seconds using a single core. By more careful choice of
elliptic curve, and focused optimization work, this number can surely
be improved.}.

On the other hand, Bitcoin's current state of 40 million unspent outputs
would grow to 100Gb and a few days of verification on a modern CPU, thanks
to Mimblewimble's use of confidential transitions. It is an open problem
how to improve this. We observe that even without cryptographic improvements,
it would go a long way to simple cap the UTXO set size as a function of
blockheight and let the fee market take care of it.
\end{itemize}

\subsection{Trust Model}

Like Bitcoin, Mimblewimble is a blockchain-based cryptocurrency intended
to be instantiated in a decentralized manner in which users may join or
leave the system at any time. Further, upon (re)joining the network, users
should be able to determine the network state, at least up to some recent
time, and verify that this state was obtained by a series of valid state
transitions, without trusting the honesty of any third parties.

However, there are two points of departure from Bitcoin's trust model:
\begin{enumerate}
\item Unlike Bitcoin, whose blockchain describes every transaction in its
entirety, and therefore allows all users to agree on and verify the precise
series of transactions that led to the current chainstate, Mimblewimble
only allows users to verify the essential features:
\begin{itemize}
\item A transaction, once committed to the block, cannot be reversed
without rewriting the block (or by the owner(s) of its outputs explicitly
undoing it).
\item The current state of all coins was obtained by zero net theft or
inflation: there are exactly as many coins in circulation as there should
be, and there for each unspent output there exists a path of transactions
leading to it, all of which are committed in the chain and authorized.

Note that there may be other paths which have also been committed in the
chain, during which some transactions may have been invalid or unauthorized
or inflationary, but since a legitimate path exists, all these things must
have netted out to zero.
\end{itemize}

\item As detailed in Section \ref{proven_expected}, verifiers may see
multiple blockchains; they compute the \emph{total work} $D$ of each
chain and declare the one with the largest $D$ to be the valid one, as
in Bitcoin.

However, in Bitcoin, this value $D$ represents both the \emph{expected work}
to produce such a blockchain as well as the \emph{proven work} of the chain,
in the sense that any party who expends significantly less than $D$ work
will be unable to produce such a chain except with negligible probability.

Mimblewimble, on the other hand, separates these. The total work $D$ still
represents the expected work to produce the blockchain, and therefore
incentivizes rational actors to contribute to the most-work chain rather
than rewriting it. The proven work, on the other hand, is capped at some
fixed value independent of the length of the chain, and serves to make
forgery by irrational lucky actors prohibitively expensive.
\end{enumerate}

\section{Preliminaries}

\subsection{Cryptographic Primitives}

\paragraph{Groups.} Throughout, $\mathcal{G}_1$, $\mathcal{G}_2$ will denote
elliptic curve groups adorned with an efficiently computable bilinear pairing
$\e:\mathcal{G}_1\times\mathcal{G}_2\to\mathcal{G}_T$, with $\mathcal{G}_T$
equal to the multiplicative group of $\mathbb{F}_{q^k}$ for some prime $q$,
small positive integer $k$. We further require both $\mathcal{G}_1$ and
$\mathbb{G}_2$ to have prime order $r$. Let $G$, $H$ be fixed generators of
$\mathcal{G}_1$ whose discrete logarithms relative to each other are unknown
(\emph{i.e.} they are nothing-up-my-sleeve (NUMS) points; $\mathcal{G}_2$
does not need any canonical generators. We will make computational hardness
assumptions about these groups as needed. We write $\mathcal{Z}_r$
for the integers modulo $r$, and write $+$ for the group operation in all
groups.

All cryptographic schemes will have an implicit \textrm{GenParams}$(1^\lambda)$
phase which generates $r$, $\mathcal{G}_1$, $\mathcal{G}_2$, $\mathcal{G}_T$,
$G$ and $H$ uniformly randomly given a security parameter $\lambda$.

\subsubsection{Standard Primitives}

\begin{defn} A \emph{commitment scheme} is a pair of algorithms
$\Commit(v, r)\to\mathcal{G}$, $\Open(v, r, C)\to\truth$ such that
$\Open(v, r, \Commit(v, r))$ accepts for all $(v, r)$ in the domain
of $\Commit$. It must satisfy the following security properties.
\begin{itemize}
\item \emph{Binding.} The scheme is \emph{binding} if for all $(v, r)$ in the
domain of $\Commit$, there is no $(v',r')\neq(v,r)$ such that
$\Open(v', r', \Commit(v, r))$ accepts.
It is \emph{computationally binding} if no PPT algorithm can produce such a
$(v', r')$ with nonneglible probability.
\item \emph{Hiding.} The scheme is \emph{(perfectly, computationally) hiding}
if the distribution of $\mathrm{Commit}(v, r)$ for uniformly random $r$ is
(equal, computationally indistinguishable) for different values of $v$.
\end{itemize}
\end{defn}
\begin{defn} We define a \emph{homomorphic commitment scheme} as one
for which there is a group operations on commitments and $\Commit$ is
homomorphic in its value parameter. That is, one where commitments to
$v$, $v'$ can be added to obtain a commitment to $v+v'$ having the
same security properties.
\end{defn}

\begin{expl} Define a \emph{Pedersen commitment} as the following scheme:
$\Commit:\mathbb{Z}_r^2\to\mathcal{G}$ maps $(v, r)$ to $vH + rG$, and
$\Open:\mathbb{Z}_r^2\times\mathcal{G}\to\truth$ checks that $(v, r)$
equals $vH + rG$.

If the discrete logarithm problem in  $\mathcal{G}$ is hard, then this
is a computationally binding, perfectly hiding homomorphic commitment
scheme\cite{pedersen2001}.
\end{expl}

\begin{defn} Given a homomorphic encryption $C$, we define a
\emph{rangeproof} on $C$ as a cryptographic proof that the committed
value of $C$ lies in some given range $[a, b]$. We further require
that rangeproofs are zero-knowledge proofs of knowledge (zkPoK) of
the opening information of the commitments.
\end{defn}
Unless otherwise stated, rangeproofs will commit to the range
$[0,2^n]$ where $n$ is small enough that no practical amount
of commitments can be summed to produce an overflow.

We may use, for example, the rangeproofs described in \cite{maxwell2015}
for Pedersen commitments, which satisfy these criteria.

\subsubsection{Sinking Signatures}

This brings us to the first new primitive needed for Mimblewimble.

\begin{defn} We define a \emph{sinking signature} as the following
collection of algorithms:
\begin{itemize}
\item $\Setup(1^\lambda)$ outputs a keypair $(\sk, \pk)$;
\item $\Sign(\sk, h)$ takes a secret key $\sk$ and nonnegative integer
``height'' $h$ which is polynomial in $\lambda$, and outputs a signature $s$.
\item $\Verify(\pk, h, s)$ takes a public key $\pk$ and signature $s$ and
outputs from $\truth$.
\end{itemize}
satisfying the following security and correctness properties:
\begin{itemize}
\item Correctness. For all polynomial $h$, $(\sk, \pk)\gets\Setup(1^\lambda)$,
$s\gets\Sign(\sk, h)$, we have that $\Verify(\pk, h, s)$ accepts.
\item Security. Let $(\cdot, \pk)\gets\Setup(1^\lambda)$. Then given $\pk$ and
an oracle $H$ which given $h$ returns $\Sign(\sk, h)$, no PPT algorithm
$\mathcal{A}$ can produce a pair $(s, h')$ with $h'>h$ for all $h$ given to
the oracle, and $\Verify(\pk, h', s)$ accepts.
(Except with negligible probability.)
\end{itemize}\label{sinkingsig}
\end{defn}

The name ``sinking signature'' is motivated by the fact that given a signature
$s$ on height $h)$, it may be possible for a forger to create a signature $s'$
on height $h'$ with the same public key and $h' \leq h$, thus ``decreasing the
height'' of the signature. We will use this feature later to aid scalability for
a Mimblewimble chain.

\begin{defn} We say a sinking signature is \emph{aggregatable} or \emph{summable}
if given a a linear combination $\pk$ of $\pk_i$ values computed from $\Setup(1^\lambda)$,
the same linear combination of $s_i\gets\Sign(\sk_i, h)$ (for fixed $h$) it
is possible to compute a signature $s$ such that $\Verify(\pk, h, s)$ accepts.

For a summable sinking signature, we generalize the above security game to allow
the adversary $\mathcal{A}$ to play polynomially many times in parallel and win
with an arbitrary linear combination of its received public keys. (It must also
provide the coefficients of the linear combination.)
\label{homodef}
\end{defn}

We propose the following summable sinking signature (Poelstra, Kulkarni).
Let $\hash$ be a random oracle hash with values in $\mathcal{G}_2$.
\begin{itemize}
\item $\Setup(1^\lambda)$ chooses a uniformly random $\sk\gets\mathbb{Z}_r$
and sets $\pk = \sk\cdot G$.
\item $\Sign(\sk, x)$ first computes the sequence $\{x_0,\ldots,x_n\}$
where $x_0 = x$ and $x_{i+1}$ is obtained by subtracting from $x_i$ the
largest power of 2 that divides it (\emph{i.e.}, by clearing its least
significant 1 bit). We observe that $x_n = 0$ and that $n$ is one plus
the number of one bits in $x$ and is therefore $O(\log_2x)$.

Next, it sets $s = \{\sk\cdot \hash(x_i\}_{i=0}^n$.
\item $\Verify(\pk, x, p)$ computes $P$ as the sum of all elements of $p$,
computes $H = \sum_{i=0}^n \hash(x_i)$, and checks that $e(G, P) = e(\pk, H)$.
\end{itemize}
Observe that the verification step uses only the sum $P$ of the elements
of the proof. However, the extra data will be useful later, so we state
it here so we can prove the scheme is secure with it.

Correctness and summability of the scheme are immediate.

\begin{thrm} This is a secure summable sinking signature, in the following
sense: if an adversary $\mathcal{A}$ exists which wins the game described
in Definition \ref{homodef}, a simulator $\mathcal{B}$ exists which can solve
the computational co-Diffie-Hellman (co-CDH) problem for $(\mathcal{G}_2,
\mathcal{G}_1)$, given oracle access to $\mathcal{A}$.\end{thrm}
\begin{proof} $\mathcal{B}$ answers random oracle queries to $H$ and
works in the following way. (Recall that the game in Definition
\ref{homodef} is the game from Definition \ref{sinkingsig} played
in parallel.)

We suppose without loss that before making signature queries on any height $x$,
the adversary first all random oracle queries needed to verify such a signature;
similarly before producing a forgery on height $x^*$ it makes the required queries.
We then suppose that in total it requests at most $q_p$ public keys and makes
at most $q_h$ random oracle queries.

\begin{enumerate}
\item $\mathcal{B}$ receives a co-CDH challenge $(G, P, Q)$ from its challenger,
where $G, P\in\mathcal{G}_1$, $Q\in\mathcal{G}_2$, and the goal is to produce
$R\in\mathcal{G}_2$ such that $\e(P, Q) = \e(G, R)$.

\item $\mathcal{A}$ responds to the $i$th public key request by generating
a uniformly random keypair $(x_i, P_i)$ and replies with $P_i + P$.

\item $\mathcal{A}$ responds to the $j$th random oracle query by generating
a uniformly random keypair $(y_j, H_j)$ except that $H_{j^*} = Q$ for
$j^*\randgets\{1,\ldots,q_h\}$, and replying with $H_i$.

\item $\mathcal{A}$ responds to the $k$th signature query on height $h^k$
and pubkey $P^k$ as follows: it computes the sequence $\{h^k_n\}$ and checks
if any of the $H(h^k_n)$ values is $Q$; if so, it aborts.

Otherwise, for each $i\{0,\ldots,n\}$ it knows a $z_i$ such that $H(h^k_i) = z_iG$
and produces $s = \{ z_iP \}_i$.

\item Finally, $\mathcal{B}$ wins by producing a forgery consisting of coefficients
$\{ c_i\}_{i=1}^m$ with not all $c_i$ zero, a pubkey $P^* = \sum_{i=1}^m c_i(P_i + P)$,
a height $h^*$ greater than any $h^*$ thus queried on, and a signature $s^* = \{S_i\}$
such that $\sum_{i=0}^{n^*}e(G, S_i) = \sum_{i=0}^{n^*} e(P^*, H(h^*_i))$.

\item If some $H(h_{i^*}^*) = Q$, then for the above sum to hold, writing $x^*$
for $P^* = x^*G$, we must have
\begin{align*}
\sum_{i=0}^{n^*} S_i
    &= \sum_{i=0}^{n^*} y^*_iP^* &\text{for $y^*_i$ such that $H(h^*_i) = y^*_iG$}	\\
    &= \sum_{i=0}^{n^*} \sum_{j=1}^m [c_jx_jy^*_iG + c_jy^*_iP]	\\
    &= \sum_{i=0}^{n^*} \sum_{j=1}^m [c_jx_jH(h^*_i) + c_jy^*_iP] 	\\
    &= \sum_{j=1}^m c_j R + \sum_{\substack{i=0\\i\neq i^*}}^{n^*} \sum_{j=1}^m [c_jx_jH(h^*_i) + c_jy^*_iP]
\end{align*}
where $R$ is the solution to $\mathcal{B}$'s CDH challenge, and every other
term is computable by $\mathcal{B}$.
\end{enumerate}
TODO show we don't abort (except with probability bounded away from 1), this win condition occurs
with nonneglible probability, also let the challenger add its own pubkeys
(as long as it declares them) (in MW it hasta declare them because any
"excess" it adds is the sum of a subset of outputs)
\end{proof}

\subsection{Mimblewimble Primitives}

\begin{defn}A \emph{Mimblewimble transaction} is the following data:
\begin{itemize}
\item A list of homomorphic commitments termed the \emph{inputs},
with attached rangeproofs. Alternately, inputs may be given as
explicit amounts, in which case they are treated as homomorphic
commitments to the given amount with zero blinding factor.
\item A list of homomorphic commitments termed the \emph{outputs},
with attached rangeproofs.
\item A blockheight $x$.
\item An \emph{excess} commitment to zero, with a summable sinking
signature on blockheight $x$ with this as pubkey.
\end{itemize}
We make the further restriction on transactions that every input
within a transaction be unique.
\end{defn}

\begin{defn} We define the \emph{sum} of a transaction to be its
outputs minus its inputs, plus its excess. If the sum is zero\footnote{By
zero we mean the homomorphic commitment which commits to zero with
zero blinding factor.}, we say the transaction is \emph{valid}.
\end{defn}

\begin{defn} We define the \emph{canonical form} of a transaction
$T$ as the transaction equal to $T$ except if any input is equal
to an output, both are removed.

Notice that the canonical form of any transaction has equal sum
to the original transaction; in particular a transaction is valid
if and only if its canonical form is valid.\end{defn}

We observe that all valid transactions are noninflationary; the
total output value must be equal to the total input value.

\begin{defn} Given a finite set of transactions $\{T_i\}$ with
pairwise disjoint input sets, we define the \emph{cut-through}
of $\{T_i\}$ as the canonical form of the union of all $T_i$'s.
\end{defn}

Next, we define some terms that will allow us to treat Mimblewimble
transactions as mechanisms for the transfer of value within a blockchain.
\begin{defn} (Ownership.) We say that a party $S$ \emph{owns} a set
of transaction outputs if she knows the opening of the sum of the
outputs.\end{defn}

\begin{defn} (Sending $n$ coins.) To send $n$ coins from $S$
to $R$, $S$ produces a transaction: chooses inputs, creates uniformly
random change output(s) and a uniformly random excess, whose sum is
a commitment to $n$. $S$ sends this to $R$ along with the opening
information of the sum.
\label{send}
\end{defn}

%\begin{lmma}It is impossible for an algorithm $R$ to produce a valid
%transaction $T$ for which it owns outputs whose committed value totals
%$n$, unless it knows the opening of the sum of a transaction $T'$, where
%$T'$ is the $T$ without the outputs that $R$ knows.
%\end{lmma}
%\begin{proof}The opening of the sum of outputs is the negative of the
%opening of $T'$.\end{proof}

\begin{defn} (Receiving $n$ coins.) To receive $n$ coins, $R$ receives
a transaction $T'$ which sums to a commitment of $m\geq n$ coins, along
with opening information of this sum, and completes it to a valid
transaction $T$ by the following process:
\begin{enumerate}
\item $R$ first produces uniformly random outputs whose total committed
value is $n$, and adds these to the transaction.
\item If $m = n$, $R$ negates the sum of this transaction and adds it
to the excess of the transaction, so that the total sum is now zero.
(It also computes a summable sinking signature for the amount added,
and adds this to the original signature, so that the final excess has
a valid signature on it.)

Otherwise the transaction now has sum committing to $m - n$, so $R$ adds
a uniformly random value to excess, updates the signature, and gives the
opening information of the new sum to another recipient who can take the
remaining value.
\end{enumerate}
\label{receive}
\end{defn}
When we define a blockchain and require transaction inputs to be outputs
of earlier transactions, we will show that after this process, $R$ is
the only party who owns any of the outputs that he added.

\section{The Mimblewimble Payment System}

\subsection{Fundamental Theorem}

\begin{thrm} (Fundamental Theorem of Mimblewimble) Suppose we have
a binding, hiding homomorphic commitment scheme. Then no algorithm
$\mathcal{A}$ can win at the following game against a challenger
$\mathcal{C}$ except with negligible probability:
\begin{enumerate}
\item (Setup.) $\mathcal{C}$ computes a finite list $L$ of uniformly
random homomorphic commitments. $\mathcal{C}$ sends $L$ to $\mathcal{A}$.
\item (Challenge.) At most polynomially many times, $\mathcal{A}$
selects some (integer) linear combination $T_i$ of $L$ and requests the
opening of this combination from $\mathcal{C}$. $\mathcal{C}$ obliges.
\item (Forgery.) $\mathcal{A}$ then chooses a new linear
combination $T$ which is not a linear combination of $\{T_i\}$ and reveals
the opening information of $T$.
\end{enumerate}
\label{fundamental}
\end{thrm}
\begin{proof}
Consider the lattice $\Lambda$ of formal linear combinations generated by $L$,
that is, the set $\{ \sum_{A\in L} b_A A : b_A \in \mathbb{Z}\}$. Consider the
quotient lattice $\Lambda/\Gamma$ where $\Gamma$ is the sublattice of
$\Lambda$ generated by the queries $\{T_i\}$. We may consider every
element of $\Lambda/\Gamma$ to be a homomorphic commitment by using
some canonical representative. In particular, every element of
$\Lambda/\Gamma$ is a homomorphic commitment which satisfies the same
hiding/blinding properties that the original scheme did.

Then the projection of every $\{T_i\}$ into $\Lambda/\Gamma$ is zero,
and the projection of $T$ is nonzero. Therefore $\mathcal{A}$ has
learned no information about the projection of $T$ from its queries;
however, if $\mathcal{A}$ knows the opening of $T$ then it also knows
the opening of the projection of $T$, contradicting the hiding
property of the homomorphic commitment scheme.
\end{proof}

\subsection{The Blockchain\label{sec:bc}}

Mimblewimble consists of a \emph{blockchain}, which is a weighted-vertex
directed rooted tree of \emph{blocks} (defined below) for which the
highest-weighted path is termed the \emph{consensus history}.

\begin{defn} We define a \emph{Mimblewimble block} as the following data:
\begin{itemize}
\item A canonical transaction, whose inputs are of one of the two forms:
\begin{itemize}
\item A reference to an output of the transaction in an earlier block; or
\item An explicit (\emph{i.e.} with zero blinding factor) input restricted
by rules above those given in this paper\footnote{A typical rule would be
that each block can have only a single \emph{coinbase input} of fixed
value.}.
\end{itemize}
\item A \emph{block header}, which has a binding commitment to earlier
blocks in the chain, termed \emph{backlinks}; the transaction included in
this block; the cut-through of this transaction and every previous block's
transactions.
\end{itemize}
If the block's transaction is valid, we say the block is \emph{valid}.
\end{defn}

In Section \ref{sec:consensus}, we will describe how blocks are weighted,
and which previous blocks specifically should be linked to. For now we
will 

\begin{defn} We define the \emph{consensus chain state} of a Mimblewimble
blockchain as the cut-through of all transactions on the consensus history.

If the consensus chain state is valid, we call the blockchain \emph{valid}.
\end{defn}
Note that for a blockchain to be valid, it is not required that all blocks
be valid, only that the entire chain sum to a valid transaction.

Next, we prove that Mimblewimble is a sound payment system, in the sense
of the following two theorems.
\begin{thrm} (No inflation.) The total value committed by the outputs of
the consensus chainstate of a valid blockchain is equal to the value of
the explicit inputs in each block.
\end{thrm}
\begin{proof} By hypothesis the consensus chain state, which is the canonical
form of the cut-through of all transactions, is valid. Since every non-explicit
input of every block is the output of a previous transaction, it does not appear
in this canonical form. Therefore the only inputs of the chain state are the
explicit ones.
\end{proof}

\begin{lmma} (Unique ownership.) Suppose that all outputs of a transaction
were created by receiving coins as in Definition \ref{receive}, and that all
blinding factors are kept secret with the specific exception of sending
coins as in Definition \ref{send}. Then for every subset of outputs in which
not all have not been sent, the only owner of that subset is the person
who created all its outputs. (In particular, if the subset contains outputs
created by different parties, then that subset has no owner.)
\label{unique}
\end{lmma}
\begin{proof} Let $O$ be an output in the subset which has not been sent.
Then the only combination of outputs containing $O$ whose commitment included
$O$ that may have been revealed also included some uniformly random excess
$E$ which was chosen when $O$ was created.

Further no other sum containing $E$ was ever revealed, so that any combination
including $O$ but not $E$ is \emph{not} a linear combination of combinations
whose opening information has been revealed. The subset in question does not
contain $E$, since $E$ is an excess not an output. Therefore by Theorem
\ref{fundamental} nobody knows the opening information of the subset except
the person who created $O$.
\end{proof}

\begin{thrm} (No theft.) Consider a valid blockchain. An output $x$
created as in Definition \ref{receive} cannot be removed from the
consensus chain state without replacing the block in which it appeared,
\emph{i.e.}, forming a higher-weighted valid blockchain not containing
this block, except by parties who (collectively) own a set $U$ of
outputs containing $x$.
\end{thrm}
\begin{proof} Suppose otherwise; then there exists a higher-weighted
chain containing the block $B$ in which $x$ appeared, but for which
the consensus chain state does not contain $x$.

Consider the transaction $T$ which is the canonical form of the
cut-through of the first block after $B$ to the tip of the chain.
(Note that $T$ may not be valid; we know only that the full chain
states are valid.)

Then the outputs of $T$ are a subset of the outputs of the new
chain state; in particular they contain rangeproofs which are
proofs of knowledge of the openings of the outputs. Similarly, the
excess value is also known. We conclude that the parties who created
these blocks (and therefore $T$) know the openings of all outputs and
sum of the excess value, and therefore own the set of all inputs of $T$.

However, the inputs of $T$ form a set of outputs containing $x$,
completing the proof.
\end{proof}

\subsection{Consensus\label{sec:consensus}}

Mimblewimble uses a hashcash\cite{back2002} style blockchain in which every
block in a blockchain is labelled by a weight called its \emph{difficulty}.
A blockchain is valid if every block of difficulty $D$ has a header which
hashes into a range of size $1/D$ of the total space of hashes. We define
a directed edge from block $A$ to $B$ iff $A$ commits to $B$ in its header,
and require each block commit to its unique \emph{parent}.

We can then refine our definition in Section \ref{sec:bc} of \emph{consensus
history} as the highest-weighted path terminating at the root. This is the
same as Bitcoin's design.

\subsubsection{Block Headers}

However, in we also define a \emph{second} graph structure on blocks as
vertices, called the \emph{compact blockchain}. We define the compact
blockchain iteratively as follows.
\begin{enumerate}
\item The genesis block is in the compact blockchain.
\item The first block after the genesis is added to the compact blockchain,
and is assigned \emph{effective difficulty} equal to its difficulty.
\item Each block added to the tip of the compact blockchain may cause
blocks to be removed from the compact chain as follows: 
\begin{enumerate}
\item First, its effective difficulty is calculated as follows. Consider
the ``maximum possible difficulty'' $M$ of the block, which is the size
of the hash space divided by the hash of the block. (This may be larger
than the actual difficulty.)

The effective difficulty is determined by starting with the block's
difficulty, then adding the effective difficulties of as many
consecutive blocks as possible, starting from the tip, so that the total
is less than or equal to $M$.

\item All blocks whose effective difficulty was used in the above calculation,
except the new block itself, are dropped from the compact chain.
\end{enumerate}
\end{enumerate}
The compact blockchain is encoded in the real blockchain by having every
block commit to a merkle sum tree (with effective difficulty the quantity
being summed) of all blocks in the current compact chain. 

We next prove several theorems to give an intuition of the properties
of the compact blockchain.

\begin{lmma} The expected work required to produce a block with effective
difficulty $D$ is equivalent to computing $D$ hashes; similarly to
produce several blocks with total effective difficulty $D'$ one must
do expected work of computing $D'$ hashes.\label{pow}\end{lmma}
\begin{proof}This is immediate in the random oracle model.\end{proof}

\begin{thrm} The expected amount of work to replace a block $B$ in the
compact chain (\emph{i.e.} produce a blockchain of greater or equal
total difficulty whose compact chain does not contain $B$) is greater
than or equal to the work needed to replace $B$, its parent in the
non-compact chain, its parent, and so on, up to but not including $B$'s
parent in the compact chain.\end{thrm}
\begin{proof} This follows immediately from Lemma \ref{pow} and the
fact that the effective difficulty of every block is defined to be
greater than or equal to the sum of the difficulty of the skipped
blocks.
\end{proof}

\begin{corl} The expected work required to produce a compact blockchain
is at least as large as the expected work required to produce a full
chain containing the same blocks.\end{corl}

\begin{thrm} Assuming constant difficulty, given a blockchain of length
$N$, the expected length of the compact chain will be $O(\log N)$.
\end{thrm}
\begin{proof} The compact chain has been defined such that the proof in
CITE, APPENDIX A still holds. We summarize it here:
\begin{enumerate}
\item First, consider starting from the tip and scanning backward until
we find a block that can skip all remaining blocks back to the genesis.
By construction this block will be in the compact chain. The probability
that such a block exists within the first $x$ blocks we check is
\[ 1 - \prod_{i=1}^x \frac{N-i}{N-i+1} = 1 - \frac{N-x}{N} = \frac{x}N \]
and the expectation of this over all $1\leq x \leq N$ is $\frac{N+1}2$,
\emph{i.e.} we expect the chain length to be halved by this one skip.
\item Inductively, we can scan back from the tip until we find a block
that skips back to the block in the previous step. This halves (in
expectation) the remaining chain, and so on. 
\end{enumerate}
The number of times we repeat this process until we have no more blocks
to skip is the length of the compact chain, since each step added one
more block to the compact chain, and since each step halved the number
of remaining blocks, we see that there are only logarithmically many
steps.
\end{proof}
We observe that small variations in difficulty do not affect the character
of this proof, and therefore the constant-difficulty case can be considered
a good approximation to the real-world situation.

\begin{thrm} for a given blockchain, there is exactly one compact chain, and it
can be verified with log-much work. further, any block which appears in the
full chain but not the compact chain, will not appear in the compact chain
of any extension of the full chain (i.e. once a block is dropped it can be
forgotten).\label{cc_unique}\end{thrm}
\begin{proof}
TODO TODO
\end{proof}

\subsubsection{Proven and Expected Work\label{proven_expected}}

However, while the expected work can be computed to be the same, the
the compact chain \textbf{does not}, in general, prove as much work as
the full chain. For example, consider a blockchain of total difficulty
$D$ across $n$ blocks, whose compact chain has $\log n$ blocks.

Suppose an attacker attempts to produce this chain in $\epsilon D$ work,
where $0<\epsilon<1$. Then this requires, on average, that each individual
block be produced in $\epsilon$ the expected time. Each block's production
time is an independent variable, so the Chernoff bound lets us approximate
this more precisely: if $\epsilon < 1$ then the probability decays
exponentially with the number of blocks in the chain.

For the full chain this means probability $O(\exp((1 - \epsilon)n))$; for
the compact chain $O(\exp((1 - \epsilon)\log n))$ or $O(n^{1-\epsilon})$.
In fact, the extreme case is even worse: a compact chain may consist of
only a single block which has difficulty $D$, in which case the probability
is simply $O(\exp(1 - \epsilon))$. This means an attacker willing to expend
some fixed percentage of the total chain work has the same probability of
successfully rewriting the chain regardless of the length of the chain.

To be conservative, we conclude that \textbf{compact chains, as described
in this paper, prove no work}\footnote{This says nothing about the compact
SPV proofs described in, e.g. CITE PROOFS OF POW, which put lower bounds
on the length of compact proofs in order to upper-bound the probability
a less-than-expected-work attacker can succeed. We cannot take this approach
because we are using these proofs in consensus code and therefore need
Theorem \ref{cc_unique}.}.

So what good are they? In Mimblewimble, we expect verifiers of a chain to
demand all blocks from the most recent two months, say, and a compact chain
from there to the start (in Section \ref{ss_cc} we will see how full verification
can proceed using only a compact chain). The resulting composite will:
\begin{itemize}
\item be forgeable with expected work equal to the entire chain's work; but
\item only \emph{prove} the most recent two months of work
\end{itemize}
Unlike Bitcoin, where the expected work to forge a blockchain is the same
(asymptotically) to its proven work, in Mimblewimble these quantities are
different. The expected work affects incentives: rational actors will choose
to extend the most-work chain rather than attempting forgeries, just as in
Bitcoin; on the other hand, the proven work affects verifiers' certainty
about the state of the world: they know at least two months worth of work
has been done to produce the chain they see, that it was not an accident,
and that if it is a forgery it was a very expensive one that cannot be
reliably repeated.

\subsubsection{Sinking Signatures and Compact Chains\label{ss_cc}}

In this section, we describe how sinking signatures interact with compact
chains, and in particular we find that it is possible to do a full
Mimblewimble verification with only $\log^2$ of the total historic block
data.

\section{Extensions\label{sec:ext}}

TODO
\subsection{Multisignature Outputs}

\subsection{Payment Channels}

\nolinenumbers

\clearpage
\bibliographystyle{amsalpha}
\bibliography{asp}

\end{document}

